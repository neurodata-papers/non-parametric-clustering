
%%%%%%%% insert above Conclusion???

An implementation of unweighted $k$-groups is available in the
energy package for R \cite{energy}.

%%%%%%%%%  here is the citation for the package, update submitted to CRAN on Aug. 11


  Maria Rizzo and Gabor Szekely (2018). energy: E-Statistics: Multivariate Inference via the Energy of
  Data. R package version 1.7-5. https://CRAN.R-project.org/package=energy

or 

@Manual{energy,
    title = {energy: E-Statistics: Multivariate Inference via the Energy of Data},
    author = {Maria Rizzo and Gabor Szekely},
    year = {2018},
    note = {R package version 1.7-5},
    url = {https://CRAN.R-project.org/package=energy},
  }

%%%%%%%%%%% bio sketch for Maria inserted below %%%%%%%%%%%%

% biography section
%
% If you have an EPS/PDF photo (graphicx package needed) extra braces are
% needed around the contents of the optional argument to biography to prevent
% the LaTeX parser from getting confused when it sees the complicated
% \includegraphics command within an optional argument. (You could create
% your own custom macro containing the \includegraphics command to make things
% simpler here.)
%\begin{IEEEbiography}[{\includegraphics[width=1in,height=1.25in,clip,keepaspectratio]{mshell}}]{Michael Shell}
% or if you just want to reserve a space for a photo:

\begin{IEEEbiography}[{\includegraphics[width=1in,clip,keepaspectratio]{gui2.png}}]{Guilherme Fran\c ca}
has a BSc in physics, and received both his MSc and
PhD in Theoretical Physics from the Institute of
Theoretical Physics IFT-UNESP/ICTP-SAIFR (2012). He was a Postdoctoral Fellow
with the High Energy Theory group in the Physics Department
at Cornell University (2013--2015), working on topics of mathematical physics
and non-perturbative methods in quantum field theory
and statistical mechanics.
He was a Postdoctoral Fellow in
the Computer Science Department at Boston College (2015), working in
optimization theory.
Since 2016 he is a Postdoctoral Fellow at Johns Hopkins University, currently
with the Mathematical Institute for Data Science (MINDS).
His current research interests are in theoretical methods in
machine learning
and optimization.
\end{IEEEbiography}

\begin{IEEEbiography}[{\includegraphics[width=1in,clip,keepaspectratio]{Maria_Rizzo.jpg}}]{Maria Rizzo}
I hold a BS and MA degree in Mathematics from the University of Toledo, a MS in Applied Statistics
from BGSU, and a PhD from BGSU in 2002. After four years as Assistant Professor at Ohio University
Department of Mathematics, I returned to BGSU in 2006, where I am currently Professor of Statistics in
the Department of Mathematics and Statistics. Primary research interests are centered on energy statistics,
distance correlation and its applications, computational statistics, and related topics. In addition
to teaching statistics, data science, and actuarial science, I enjoy writing textbooks about
statistical computing and using R software, a forthcoming research monograph on energy statistics 
(joint with G. Szekely) and have other book writing projects under contract as well. 
\end{IEEEbiography}

% insert where needed to balance the two columns on the last page with
% biographies
%\newpage

\begin{IEEEbiography}[{\includegraphics[width=1in,clip,keepaspectratio]{jovo.png}}]{Joshua~T.~Vogelstein}
I received a B.S degree from the Department of Biomedical Engineering (BME) at Washington University in St. Louis, MO in 2002, a M.S. degree from the Department of Applied Mathematics and Statistics (AMS) at Johns Hopkins University (JHU) in Baltimore, MD in 2009, and a Ph.D. degree from the Department of Neuroscience at JHU in 2009. I was a Postdoctoral Fellow in AMS@JHU from 2009 until 2011, at which time I was appointed an Assistant Research Scientist, and became a member of the Institute for Data Intensive Science and Engineering. I spent 2 years at Information Initiative at Duke University, before coming home to my current appointment as Assistant Professor in BME@JHU, and core faculty in both the Institute for Computational Medicine and the Center for Imaging Science, as well as a member of the Kavli Neuroscience Discovery Institute. I married my kindergarten sweetheart in the summer of 2014.

My primary research interest is to extend and fuse statistical machine learning and big data science to address the most important brain science and mental health questions of our time, particularly  connectomics. Our group’s research has been featured in a number of prominent scientific and engineering journals and conferences including Nature, Nature Methods, Science, Cell, PNAS, Science Translational Medicine, JMLR, PAMI, Annals of Applied Statistics,  NIPS, and SIAM Journal of Matrix Analysis and Applications. In 2011, I co-founded the Open Connectome Project which expanded in 2015 to be NeuroData, whose mission is to enable terascale neuroscience for everyone. All our work is conducted according to the highest standards of open science.
\end{IEEEbiography}

% You can push biographies down or up by placing
% a \vfill before or after them. The appropriate
% use of \vfill depends on what kind of text is
% on the last page and whether or not the columns
% are being equalized.

%\vfill

% Can be used to pull up biographies so that the bottom of the last one
% is flush with the other column.
%\enlargethispage{-5in}



% that's all folks
\end{document}


